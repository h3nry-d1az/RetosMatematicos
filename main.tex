\def\jobname{main} %
\documentclass[
    fecha={5 de agosto de 2025},palabrasclave={RetoSecundaria, ago2025, álgebra, dif1},codigo=minted
    ]{RetoMatematico}

\usepackage{RetoExtra} % Comandos y paquetes adicionales

\begin{document}
\maketitle
\thispagestyle{empty}

\ejercicio{
	\begin{enumerate}[label=\emph{\alph*})]
		\item Demuéstrese la siguiente identidad: \[\l(x^2+y^2+\l(x+y\r)^2\r)^2 = 2\l(x^4+y^4+\l(x+y\r)^4\r),\quad\forall x, y \in \R.\]
		\item Como aplicación del resultado anterior, calcúlese el valor exacto de \[\sqrt{\frac{23^4+87^4+110^4}{2}}.\]
	\end{enumerate}
}{Propuesto por Antonio Roberto Martínez Fernández.}

\forma
\begin{enumerate}[label=\emph{\alph*})]
\item La fórmula propuesta en el enunciado es la conocida como \emph{identidad de Candido} y se puede comprobar desarrollando cada miembro por separado y viendo que coinciden. En efecto, se tiene que
\begin{align*}
	\Bigl(x^2&+y^2+\l(x+y\r)^2\Bigr)^2 = \l(x^2 + y^2\r)^2 + (x+y)^4 + 2\l(x^2 + y^2\r)(x+y)^2\\
	&= \underbrace{x^4 + y^4 + 2x^2y^2}+\underbrace{x^4 + 4x^3y + 6x^2y^2 + 4xy^3 + y^4} + 2\l(x^2 + y^2\r)(x+y)^2\\
	&= 2x^4 + 2y^4 + 8x^2y^2 + 4x^3y + 4xy^3 + 2\l(x^4 + x^2y^2 + 2x^3 y + x^2 y^2 + y^4 + 2xy^3\r)\\
	&= 2(2x^4 + 2y^4 + 6x^2y^2 + 4x^3 y + 4xy^3)\\
	&= 2(x^4 + y^4 + \underbrace{x^4 + 4x^3 y + 6x^2y^2 + 4xy^3 + y^4})\\
	&= 2\l(x^4 + y^4 + (x+y)^4\r). \tag*{\qedsymbol}
\end{align*}

\item Aplicando la identidad anterior con $x = 23$ e $y = 87$, todo se reduce a calcular los
cuadrados de 23, 87 y de $23 + 87 = 110$ y echar unas sencillas cuentas que pueden hacerse
a mano: \begin{align*}
	\sqrt{\frac{23^4+87^4+110^4}{2}} &= \sqrt{\frac{2\cdot\l(23^4+87^4+110^4\r)}{4}} = \frac{\sqrt{\l(23^2+87^2+110^2\r)^2}}{2}\\
	&= \frac{529 + 7\ 569 + 12\ 100}{2} = \frac{20\ 198}{2} = 10\ 099.
\end{align*} Por lo tanto \enfasis{\sqrt{\frac{23^4+87^4+110^4}{2}} = 10\ 099}
\end{enumerate}

\forma
\begin{enumerate}[label=\emph{\alph*})]
\item Este resultado puede demostrarse fácilmente expandiendo ambas expresiones y verificando que son idénticas: \begin{align*}
    \left(x^2 + y^2 + (x+y)^2\right)^2 &= \left(2x^2 + 2y^2 + 2xy\right)^2 = 4x^4 + 4y^4 + 12x^2y^2 + 8x^3y + 8xy^3\\
    &= 2(x^4 + y^4 + (x^4 + 4x^3 y + 6x^2y^2 + 4 xy^3 + y^4))\\
	&= 2(x^4 + y^4 + (x+y)^4).
\end{align*}
\item Y el segundo apartado no es sino un corolario de lo anterior: \begin{align*}
    \sqrt{\frac{23^4 + 87^4 + 110^4}{2}} &= \sqrt{\frac{2(23^4 + 87^4 + (23+87)^4)}{4}} \stackrel{\text{a)}}{=} \sqrt{\frac{\left(23^2 + 87^2 + 110^2\right)^2}{4}}\\
	&= \frac{23^2 + 87^2 + 110^2}{2} = \frac{(2\cdot 11 + 1)^2 + (2\cdot 43 + 1)^2 + (2\cdot 55)^2}{2}\\
	&= 2(11^2 + 43^2 + 55^2) + 109.
\end{align*} Haciendo estos últimos cómputos a mano (no son sino productos de números relativamente pequeños) se llega a 10099 como resultado final.
\end{enumerate}

\nseccion{Método por ordenador}
Con el siguiente código de \mathematica, digo Python, podemos comprobar que en efecto es cierto para $x,y\leq 10\ 000$, y como ese número es muy grande, el teorema ha de ser cierto por narices:
\begin{codigo}{python}
def main() -> None:
	for x in range(1, 10_001):
		for y in range(1, 10_001):
			assert (x**2 + y**2 + (x+y)**2)**2\
			    == 2*(x**4 + y**4 + (x+y)**4)

if __name__ == '__main__': main()
\end{codigo}

\nseccion{Texto de ejemplo}
Aquí va un poco de <<lorem ipsum>> para que los resolutores no se queden solos en esta nueva página:

\begin{wrapfigure}{r}{36mm}
		\psset{unit=3cm,algebraic=true,dimen=middle,dotstyle=o,dotsize=4pt 0,linewidth=0.6pt,arrowsize=3pt 2,arrowinset=0.25}
		\centering
		\vspace{-4.5mm}
		\begin{pspicture*}(-0.13,-0.12)(1.13,1.13)
		  \pstGeonode[PointName=none,PointSymbol=none]
			(0,0){A}
			(0,1){B}
			(1,1){C}
			(1,0){D}
			(0.41,1){P}
			\pstInterLL[PointName=none,PointSymbol=none]{A}{P}{B}{D}{E}
			\pspolygon[fillcolor=verdeodi,fillstyle=solid,opacity=0.15](B)(D)(A)(P)
			\pspolygon(A)(B)(C)(D)
			\rput(A){\uput{0.6ex}[-135](0,0){\(A\)}}
			\rput(B){\uput{0.6ex}[135](0,0){\(B\)}}
			\rput(C){\uput{0.6ex}[45](0,0){\(C\)}}
			\rput(D){\uput{0.6ex}[-45](0,0){\(D\)}}
			\rput(P){\uput{0.6ex}[90](0,0){\(P\)}}
			\rput(E){\uput{0.6ex}[10](0,0){\(E\)}}
		\end{pspicture*}
        \caption{}
	\end{wrapfigure}
\lipsum[1-4]

\resolutores{Henry Díaz Bordón, José Manuel Sánchez Muñoz, Pablo Vitoria García y el proponente.}

\end{document}
